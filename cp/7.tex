\documentclass{article}

\usepackage{amsmath}
\usepackage[russian]{babel}
\usepackage[margin=2.5cm]{geometry}
\usepackage{booktabs}

\usepackage{tikz}
\usetikzlibrary{graphs, babel, quotes, calc, arrows.meta}

\title{Седьмая задача}
\author{Векнков К. С. -- М8О-105Б-23 -- 7 вариант}
\date{Май, 2024}

\begin{document}
\maketitle

\section*{Дано}
\begin{center}
    \begin{tikzpicture}[->, circ/.style={circle, draw}]
        \path[nodes={circ}]
            (0, 3) node(1) {$v_1$}
            (2, 5) node(2) {$v_2$}
            (4, 6) node(3) {$v_3$}
            (6, 5) node(4) {$v_4$}
            (4, 3) node(5) {$v_5$}
            (2, 1) node(6) {$v_6$}
            (4, 0) node(7) {$v_7$}
            (6, 1) node(8) {$v_8$}
            (8, 3) node(9) {$v_9$}
        ;
        \draw[-, nodes={circle, fill=white}, thick] (1) -- node{4} (2);
        \draw[-, nodes={circle, fill=white}, thick] (1) -- node{8} (5);
        \draw[-, nodes={circle, fill=white}, thick] (1) to[in=-120, out=20] node{9} (3);
        \draw[-, nodes={circle, fill=white}, thick] (1) -- node{3} (6);
        \draw[-, nodes={circle, fill=white}, thick] (1) to[in=120, out=-20] node{8} (7);

        \draw[-, nodes={circle, fill=white}, thick] (2) -- node{4} (3);
        \draw[-, nodes={circle, fill=white}, thick] (2) -- node{4} (5);
        \draw[-, nodes={circle, fill=white}, thick] (3) -- node{8} (4);

        \draw[-, nodes={circle, fill=white}, thick] (4) -- node{12} (9);
        \draw[-, nodes={circle, fill=white}, thick] (5) -- node{5} (4);
        \draw[-, nodes={circle, fill=white}, thick] (5) -- node{7} (9);
        \draw[-, nodes={circle, fill=white}, thick] (5) -- node{5} (8);
        \draw[-, nodes={circle, fill=white}, thick] (6) -- node{5} (5);
        \draw[-, nodes={circle, fill=white}, thick] (6) -- node{3} (7);

        \draw[-, nodes={circle, fill=white}, thick] (7) -- node{8} (8);
        \draw[-, nodes={circle, fill=white}, thick] (8) -- node{13} (9);
        ;
    \end{tikzpicture}
\end{center}

\section*{Задание}
Пострить максимальный поток по транспортной сети


\section*{Решение}

\begin{center}
    \begin{tikzpicture}[->, circ/.style={circle, draw}]
        \path[nodes={circ}]
            (0, 3) node(1) {$v_1$}
            (2, 5) node(2) {$v_2$}
            (4, 6) node(3) {$v_3$}
            (6, 5) node(4) {$v_4$}
            (4, 3) node(5) {$v_5$}
            (2, 1) node(6) {$v_6$}
            (4, 0) node(7) {$v_7$}
            (6, 1) node(8) {$v_8$}
            (8, 3) node(9) {$v_9$}
        ;
        \draw[->, nodes={circle, fill=white}, thick] (1) -- node{0+4} (2);
        \draw[->, nodes={circle, fill=white}, thick] (1) -- node{0+1+7} (5);
        \draw[-, nodes={circle, fill=white}, thick] (1) to[in=-120, out=20] node{0+4} (3);
        \draw[->, nodes={circle, fill=white}, thick] (1) -- node{0+3} (6);
        \draw[-, nodes={circle, fill=white}, thick] (1) to[in=120, out=-20] node{0+5} (7);

        \draw[->, nodes={circle, fill=white}, thick] (2) -- node{0+4} (3);
        \draw[-, nodes={circle, fill=white}, thick] (2) -- node{0} (5);
        \draw[->, nodes={circle, fill=white}, thick] (3) -- node{0+4+4} (4);

        \draw[->, nodes={circle, fill=white}, thick] (4) -- node{0+4+4} (9);
        \draw[-, nodes={circle, fill=white}, thick] (5) -- node{0} (4);
        \draw[->, nodes={circle, fill=white}, thick] (5) -- node{0+7} (9);
        \draw[-, nodes={circle, fill=white}, thick] (5) -- node{0+1} (8);
        \draw[-, nodes={circle, fill=white}, thick] (6) -- node{0} (5);
        \draw[->, nodes={circle, fill=white}, thick] (6) -- node{0+3} (7);

        \draw[->, nodes={circle, fill=white}, thick] (7) -- node{0+3+5} (8);
        \draw[-, nodes={circle, fill=white}, thick] (8) -- node{0+3+5+1} (9);
        ;
    \end{tikzpicture}
\end{center}

\begin{enumerate}
    \item $v_1 - v_2 - v_3 - v_4 - v_9$: $\\ \min\{4, 4, 8, 12\} = 4$
    \item $v_1 - v_6 - v_7 - v_8 - v_9$: $\\ \min\{3, 3, 8, 14\} = 3$
    \item $v_1 - v_3 - v_4 - v_9$: $\\ \min\{9, 8 - 4, 12 - 4\} = 4$
    \item $v_1 - v_5 - v_9$: $\\ \min\{8, 7\} = 7$
    \item $v_1 - v_7 - v_8 - v_9$: $\\ \min\{8, 8 - 3, 14 - 3\} = 5$
    \item $v_1 - v_5 - v_8 - v_9$: $\\ \min\{8 - 7, 5, 14 - 8\} = 1$\
\end{enumerate}\\

Величина полного потока $\Phi_1 = 9 + 7 + 8 = 24$
\begin{center}
    \begin{tikzpicture}[->, circ/.style={circle, draw}]
        \path[nodes={circ}]
            (0, 3) node(1) {$v_1$}
            (2, 5) node(2) {$v_2$}
            (4, 6) node(3) {$v_3$}
            (6, 5) node(4) {$v_4$}
            (4, 3) node(5) {$v_5$}
            (2, 1) node(6) {$v_6$}
            (4, 0) node(7) {$v_7$}
            (6, 1) node(8) {$v_8$}
            (8, 3) node(9) {$v_9$}
        ;
        \draw[->, nodes={circle, fill=white}, thick] (1) -- node{4} (2);
        \draw[->, nodes={circle, fill=white}, thick] (1) -- node{8} (5);
        \draw[->, nodes={circle, fill=white}, thick] (1) to[in=-120, out=20] node{4+4} (3);
        \draw[->, nodes={circle, fill=white}, thick] (1) -- node{3} (6);
        \draw[-, nodes={circle, fill=white}, thick] (1) to[in=-120, out=-90] node{5+3} (7);

        \draw[->, nodes={circle, fill=white}, thick] (2) -- node{4-4} (3);
        \draw[-, nodes={circle, fill=white}, thick] (2) -- node{1} (5);
        \draw[->, nodes={circle, fill=white}, thick] (3) -- node{8} (4);

        \draw[->, nodes={circle, fill=white}, thick] (4) -- node{8+4} (9);
        \draw[-, nodes={circle, fill=white}, thick] (5) -- node{0+4} (4);
        \draw[->, nodes={circle, fill=white}, thick] (5) -- node{7} (9);
        \draw[-, nodes={circle, fill=white}, thick] (5) -- node{1+3} (8);
        \draw[-, nodes={circle, fill=white}, thick] (6) -- node{0+3} (5);
        \draw[->, nodes={circle, fill=white}, thick] (6) -- node{3-3} (7);

        \draw[->, nodes={circle, fill=white}, thick] (7) -- node{8} (8);
        \draw[-, nodes={circle, fill=white}, thick] (8) -- node{9+3} (9);
    \end{tikzpicture}
\end{center}

\begin{enumerate}
    \item $v_1 - v_3 - v_2 - v_5 - v_4 - v_9$:\\ $\Delta_1 = \min\{9 - 4, 4, 4, 5, 12 - 8\} = 1$
    \item $v_1 - v_7 - v_6 - v_5 - v_8 - v_9$:\\ $\Delta_2 = \min\{8 - 5, 3, 5, 5 - 1, 14 - 9\} = 3$
\end{enumerate}\\

Величина максимального потока $\Phi = 12 + 7 + 12 = 31$

\section*{Ответ}
$\Phi = 31$


\end{document}