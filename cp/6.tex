\documentclass{article}

\usepackage{amsmath}
\usepackage[russian]{babel}
\usepackage[margin=2.5cm]{geometry}
\usepackage{booktabs}


\usepackage{tikz}
\usetikzlibrary{graphs, babel, quotes, calc, arrows.meta}
\setcounter{MaxMatrixCols}{40}

\title{Шестая задача}
\author{Векнков К. С. -- М8О-105Б-23 -- 7 вариант}
\date{Май, 2024}

\begin{document}
\maketitle

\section*{Дано}
\begin{center}
    \begin{tikzpicture}[circ/.style={circle, draw}]
        \path[nodes={circ}]
            (0, 6) node(1) {$v_1$}
            (3, 6) node(2) {$v_2$}
            (6, 6) node(3) {$v_3$}
            (0, 3) node(4) {$v_4$}
            (3, 3) node(5) {$v_5$}
            (6, 3) node(6) {$v_6$}
            (0, 0) node(7) {$v_7$}
            (3, 0) node(8) {$v_8$}
            (6, 0) node(9) {$v_9$}
        ;
        \draw[nodes={circle, fill=white}, thick]
            (1) -- node{8} (2)
            (1) -- node{1} (4)
            (2) -- node{7} (3)
            (2) -- node{10} (5)
            (3) -- node{6} (6)
            (4) -- node{9} (5)
            (4) -- node{2} (7)
            (5) -- node{11} (6)
            (5) -- node{12} (8)
            (6) -- node{5} (9)
            (7) -- node{3} (8)
            (8) -- node{4} (9)
        ;
    \end{tikzpicture}
\end{center}


\section*{Задание}
Пусть каждому ребру неориентированного графа соответствует некоторый элемент 
электрической цепи. Составить линейно независимые системы уравнений Кирхгофа для
токов и напряжений. Пусть первому и пятому ребру соответствуют источники тока с ЭДС 
E1 и E2 (полярность выбирается произвольно), а остальные элементы являются сопротивлениями.
Используя закон Ома, и, предполагая внутренние сопротивления источников тока
равными нулю, получить систему уравнений для токов


\section*{Решение}
Зададим произвольную ориентацию на графе:

\begin{center}
    \begin{tikzpicture}[circ/.style={circle, draw}]
        \path[nodes={circ}]
            (0, 6) node(1) {$v_1$}
            (3, 6) node(2) {$v_2$}
            (6, 6) node(3) {$v_3$}
            (0, 3) node(4) {$v_4$}
            (3, 3) node(5) {$v_5$}
            (6, 3) node(6) {$v_6$}
            (0, 0) node(7) {$v_7$}
            (3, 0) node(8) {$v_8$}
            (6, 0) node(9) {$v_9$}
        ;
        \draw[-{Stealth[length=7pt]}] (1) -- (2);
        \draw[-{Stealth[length=7pt]}] (2) -- (3);
        \draw[-{Stealth[length=7pt]}] (2) -- (5);
        \draw[-{Stealth[length=7pt]}] (3) -- (6);
        \draw[-{Stealth[length=7pt]}] (4) -- (1);
        \draw[-{Stealth[length=7pt]}] (5) -- (4);
        \draw[-{Stealth[length=7pt]}] (5) -- (8);
        \draw[-{Stealth[length=7pt]}] (6) -- (5);
        \draw[-{Stealth[length=7pt]}] (6) -- (9);
        \draw[-{Stealth[length=7pt]}] (7) -- (4);
        \draw[-{Stealth[length=7pt]}] (8) -- (7);
        \draw[-{Stealth[length=7pt]}] (9) -- (8);

        \draw[nodes={circle, fill=white}, thick]
            (1) -- node{$X_1$} (2)
            (1) -- node{$X_3$} (4)
            (2) -- node{$X_2$} (3)
            (2) -- node{$X_4$} (5)
            (3) -- node{$X_5$} (6)
            (4) -- node{$X_6$} (5)
            (4) -- node{$X_8$} (7)
            (5) -- node{$X_7$} (6)
            (5) -- node{$X_9$} (8)
            (6) -- node{$X_{10}$} (9)
            (7) -- node{$X_{11}$} (8)
            (8) -- node{$X_{12}$} (9)
        ;
    \end{tikzpicture}
\end{center}


Получаем цикломатическую матрицу
$$
C =
\begin{pmatrix}
%   1    2     3     4     5    6     7    8     9    10   11   12
    0 && 1 &&  0 && -1 &&  1 && 0 &&  1 && 0 &&  0 && 0 && 0 && 0 \\
    1 && 0 &&  1 &&  1 &&  0 && 1 &&  0 && 0 &&  0 && 0 && 0 && 0 \\
    1 && 0 &&  1 &&  1 &&  0 && 0 &&  0 && 1 &&  1 && 0 && 1 && 0 \\
    0 && 0 &&  0 &&  0 &&  0 && 0 && -1 && 0 && -1 && 1 && 0 && 1 \\
\end{pmatrix}
$$

По закону Кирхгофа для напряжения:
\begin{align*}
    C \times U = 0 = 
    \begin{pmatrix}
    %   1    2     3     4     5    6     7    8     9    10   11   12
        0 && 1 &&  0 && -1 &&  1 && 0 &&  1 && 0 &&  0 && 0 && 0 && 0 \\
        1 && 0 &&  1 &&  1 &&  0 && 1 &&  0 && 0 &&  0 && 0 && 0 && 0 \\
        1 && 0 &&  1 &&  1 &&  0 && 0 &&  0 && 1 &&  1 && 0 && 1 && 0 \\
        0 && 0 &&  0 &&  0 &&  0 && 0 && -1 && 0 && -1 && 1 && 0 && 1 \\
    \end{pmatrix}
    \times
    \begin{pmatrix}
        u_1 \\ \dots \\ u_{12}
    \end{pmatrix}
    \Longrightarrow \\ \Longrightarrow
    \begin{cases}
        u_2 - u_4 + u_5 + u_7 = 0 \\
        u_1 + u_3 + u_4 + u_6 = 0 \\
        u_1 + u_3 + u_4 + u_8 + u_9 + u_{11} = 0 \\
        -u_7 - u_9 + u_{10} + u_{12} = 0
    \end{cases}
\end{align*}

\begin{center}
    \begin{tikzpicture}[circ/.style={circle, draw}]
        \path[nodes={circ}]
            (0, 6) node(1) {$v_1$}
            (3, 6) node(2) {$v_2$}
            (6, 6) node(3) {$v_3$}
            (0, 3) node(4) {$v_4$}
            (3, 3) node(5) {$v_5$}
            (6, 3) node(6) {$v_6$}
            (0, 0) node(7) {$v_7$}
            (3, 0) node(8) {$v_8$}
            (6, 0) node(9) {$v_9$}
        ;
        \draw[nodes={circle, fill=white}, thick]
            (1) -- node{$E_1$} (2)
            (1) -- node{$r_2$} (4)
            (2) -- node{$r_1$} (3)
            (2) -- node{$r_3$} (5)
            (3) -- node{$E_2$} (6)
            (4) -- node{$r_4$} (5)
            (4) -- node{$r_6$} (7)
            (5) -- node{$r_5$} (6)
            (5) -- node{$r_7$} (8)
            (6) -- node{$r_8$} (9)
            (7) -- node{$r_9$} (8)
            (8) -- node{$r_{10}$} (9)
        ;
    \end{tikzpicture}
\end{center}

Подставим закон Ома:
\begin{cases}
    i_2  r_1 - i_4  r_3  + E_2 + i_7 r_6 = 0 \\
    E_1 + i_3 r_2 + i_4 r_3 + i_6 r_4 = 0 \\
    E_1 + i_3 r_2 + i_4 r_3 + i_8 r_6 + i_9 r_7 + i_{11} r_9 = 0 \\
    -i_7 r_5 - i_9 r_7 + i_{10} r_8 + i_{12} r_{10} = 0
\end{cases}\\\\\\

Составим матрицу инцидентности
$$
        B =
        \begin{pmatrix}
    %   1    2     3     4     5    6     7    8     9    10   11   12
        -1 &&  0 &&  1 &&  0 &&  0 &&  0 &&  0 &&  0 &&  0 &&  0 &&  0 &&  0 \\
        1  && -1 &&  0 && -1 &&  0 &&  0 &&  0 &&  0 &&  0 &&  0 &&  0 &&  0 \\
        0  &&  1 &&  0 &&  0 && -1 &&  0 &&  0 &&  0 &&  0 &&  0 &&  0 &&  0 \\
        0  &&  0 && -1 &&  0 &&  0 &&  1 &&  0 &&  1 &&  0 &&  0 &&  0 &&  0 \\
        0  &&  0 &&  0 &&  1 &&  0 && -1 &&  1 &&  0 && -1 &&  0 &&  0 &&  0 \\
        0  &&  0 &&  0 &&  0 &&  1 &&  0 && -1 &&  0 &&  0 && -1 &&  0 &&  0 \\
        0  &&  0 &&  0 &&  0 &&  0 &&  0 &&  0 && -1 &&  0 &&  0 &&  1 &&  0 \\
        0  &&  0 &&  0 &&  0 &&  0 &&  0 &&  0 &&  0 &&  1 &&  0 && -1 &&  1 \\
        0  &&  0 &&  0 &&  0 &&  0 &&  0 &&  0 &&  0 &&  0 &&  1 &&  0 && -1 \\
        \end{pmatrix}
$$\\

Для получения линейно независимой системы нужно исключить одну строку:\\\\
\begin{cases}
    -i_1 + i_3 = 0 \\
    i_1 - i_2 - i_4 = 0 \\
    i_2 - i_5 = 0 \\
    -i_3 + i_6 + i_8 = 0 \\
    i_5 - i_7 - i_{10} = 0 \\
    -i_8 + i_{11} = 0 \\ 
    i_9 - i_{11} + i_{12} = 0 \\
    i_{10} - i_{12} = 0
\end{cases}\\\\

Совместная система имеет вид:\\\\
\begin{cases}
    -i_1 + i_3 = 0 \\
    i_1 - i_2 - i_4 = 0 \\
    i_2 - i_5 = 0 \\
    -i_3 + i_6 + i_8 = 0 \\
    i_5 - i_7 - i_{10} = 0 \\
    -i_8 + i_{11} = 0 \\ 
    i_9 - i_{11} + i_{12} = 0 \\
    i_{10} - i_{12} = 0 \\ 
    i_2  r_1 - i_4  r_3  + E_2 + i_7 r_6 = 0 \\
    E_1 + i_3 r_2 + i_4 r_3 + i_6 r_4 = 0 \\
    E_1 + i_3 r_2 + i_4 r_3 + i_8 r_6 + i_9 r_7 + i_{11} r_9 = 0 \\
    -i_7 r_5 - i_9 r_7 + i_{10} r_8 + i_{12} r_{10} = 0
\end{cases}\\\\

12 уравнений и 12 неизвестных $i_1, i_2, i_3, i_4, i_5, i_6, i_7, i_8, i_9, i_{10}, i_{11}, i_{12}$ \\

ЭДС - $E_1, E_2$ и сопротивление $r_1, r_2, r_3, r_4, r_5, r_6, r_7, r_8, r_9, r_{10}$ известны.

\section*{Ответ}
\begin{equation*}
    \begin{cases}
        -i_1 + i_3 = 0 \\
        i_1 - i_2 - i_4 = 0 \\
        i_2 - i_5 = 0 \\
        -i_3 + i_6 + i_8 = 0 \\
        i_5 - i_7 - i_{10} = 0 \\
        -i_8 + i_{11} = 0 \\ 
        i_9 - i_{11} + i_{12} = 0 \\
        i_{10} - i_{12} = 0 \\ 
        i_2  r_1 - i_4  r_3  + E_2 + i_7 r_6 = 0 \\
        E_1 + i_3 r_2 + i_4 r_3 + i_6 r_4 = 0 \\
        E_1 + i_3 r_2 + i_4 r_3 + i_8 r_6 + i_9 r_7 + i_{11} r_9 = 0 \\
        -i_7 r_5 - i_9 r_7 + i_{10} r_8 + i_{12} r_{10} = 0
    \end{cases}
\end{equation*}

\end{document}
